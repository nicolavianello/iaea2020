\documentclass[12pt, a4paper, twoside]{article}
\usepackage{fancyhdr}
\pagestyle{fancy}
\fancyhf{}
\renewcommand{\headrulewidth}{0pt}
\addtolength{\headwidth}{\marginparsep}
\addtolength{\headwidth}{\marginparwidth}
\rhead[]{\large\textbf{EXD}}
\usepackage{fontspec}
\setmainfont{Times New Roman}
\usepackage[margin=2.5cm]{geometry}
\usepackage{amsmath,amsfonts,amsthm}
\usepackage{graphicx}
% \graphicspath{{../pdfbox/}}
%% bibliography setting nature style, footnotesize in bibliography and
%% avoiding in in articles
\usepackage[style=nature,defernumbers=true,maxnames=1,firstinits=true,uniquename=init,backend=bibtex8,arxiv=abs,mcite]{biblatex}
\bibliography{../biblio}
%\renewcommand{\bibfont}{\normalfont\footnotesize}
\renewbibmacro{in:}{%
  \ifentrytype{article}{}{%
  \printtext{\bibstring{in}\intitlepunct}}}
\DeclareFieldFormat[article]{title}{}
\AtBeginBibliography{\footnotesize}
%% to set line space in bibliography
\usepackage{setspace}
%% to add affiliation to the title
\usepackage[affil-it]{authblk}
\renewcommand\Affilfont{\itshape\small}
\setlength{\affilsep}{.2em}
%% for figure caption
\usepackage[format=plain, font=scriptsize,labelfont=bf]{caption}
%% for figure wrapping
\usepackage{wrapfig}
%% for reference in a multi col
\usepackage{multicol}
% \captionsetup[figure]{font={footnotesize, stretch=1.}, belowskip=.01pt,
%   aboveskip=0.3pt}
\captionsetup[figure]{font={footnotesize, stretch=1.}, skip=0pt}
%% titling
\makeatletter
\renewcommand{\maketitle}{\bgroup\setlength{\parindent}{0pt}
\begin{flushleft}
{\LARGE
  \textbf{\@title}}

\vspace{0.3ex}

  \@author
\end{flushleft}\egroup
}
\makeatother
\title{SOL profile and transport and relation to divertor conditions in H-Mode plasmas: a cross-machine comparison}
\author{N. Vianello$^{1}$,M. Dunne, B. Lomanowski, N. Walkden,
  M. Griener, B. Tal, W. Wolfrum, I. Cziegler, D Brida, C. Tsui,
  O. Fevrier, H. Reimerdes, C. Theiler, M. Bernert, A. Hakkola, A. Huber,
  D. Carralero$^{2, 3}$,
  V. Naulin$^{6}$,
  M. Agostini$^{1}$, J. Boedo${^5}$,
  B. Labit$^{4}$,  C. Theiler$^4$,
  D. Aguiam$^{7}$,
  S. Allan$^{8}$, M. Bernert$^{2}$,
  S. Costea$^{9}$, I. Cziegler$^{10}$,
  H. De Oliveira$^{4}$, J. Galdon-Quiroga$^{11}$,
  G. Grenfell$^{1}$, A. Hakola$^{12}$,
  C. Ionita$^{9}$, H. Isliker$^{13}$,
  A. Karpushov$^{4}$,
  J. Kovacic$^{14}$,  B. Lipschultz$^{10}$,
  R. Maurizio$^{4}$, K. McClements$^{8}$, F. Militello$^{8}$,
  J. Olsen$^{6}$, J. J. Rasmussen$^{6}$, T. Ravensbergen$^{16}$,
  H. Reimerdes$^{4}$, B. Schneider$^9$, R. Schrittwieser$^9$,
  M. Spolaore$^1$, K. Verhaegh$^{10}$, J. Vicente$^7$,
  N. Walkden$^8$, W. Zhang$^2$, E. Wolfrum$^2$, the ASDEX-Upgrade Team,
  the TCV-Team, the EUROfusion MST1 Team$^{*}$}

\affil{
  $^1$Consorzio RFX, Padova,Italy,
  $^{2}$Max-Planck-Institut f{\"u}r Plasmaphysik, Garching, Germany,
  $^{3}$CIEMAT Laboratorio Nacional de Fusi{\'o}n, Madrid, Spain,
  $^{4}$EPFL-SPC, Switzerland,
  $^5$UCSD,  La Jolla, USA,
  $^{6}$DTU,  Copenhagen, Denmark,
  $^7$IPFN, Instituto Superior T{\'e}cnico, Lisboa, Portugal,
  $^{8}$CCFE, Culham, UK,
  $^9$Institute for Ion Physics and Applied Physics,
  Innsbruck,  Austria,
  $^{10}$York Plasma Institute, University of York, UK,
  $^{11}$University of Seville, Seville Spain,
  $^{12}$VTT, Espoo, Finland,
  $^{13}$Aristotle University of Thessaloniki, Greece,
  $^{14}$Jozef Stefan Institute, Ljubljana,
  $^{16}$DIFFER—Dutch Institute for Fundamental Energy Research, Netherlands,
  $^{*}$See the author list H. Meyer et al 2017 Nucl. Fusion 57 102014}
\date{\vspace{-3.5ex}}
%%% ------------------------------------------------------------
%%% BEGIN DOCUMENT
%%% ------------------------------------------------------------
\begin{document}
\maketitle
\vspace{-1.2em}
{\it \small Corresponding Author:} {nicola.vianello@igi.cnr.it}

Plasma Wall Interaction (PWI) is a subject of intense studies
in the context of fusion energy research for the understanding of the amount of heat
loads, tritium retention, and the lifetime of different Plasma Facing
Components.
In recent years great
efforts have been devoted to the
interpretation of Scrape Off Layer (SOL) transport, with clear impact also on the design of
future machines \cite{Kocan:2015dc}.
Transport in the SOL region, resulting
from a competition between sources and losses parallel and
perpendicular to the magnetic field, is dominated by
the presence of intermittent structures, filaments, which strongly contribute to
particle and energy losses both in L- and H-mode regimes.
On the other side, with the
approaching if ITER-era it is mandatory to
address SOL transport in regimes which are relevant from the ITER
perspective. As clearly highlighted in \cite{pitts:2019}, in order to
keep the power fluxes densities acceptable for the target material
high neutral pressure and partial detachment are needed in order to
ensure maximum tolerable loads based on avoidance of W recrystallization.

Experimentally these regimes are obtained in present experiments
with high gas throughput leading to high density regimes and these
regimes are generally accompanied in L-Mode by the development of a
\emph{shoulder formation}
describing the progressive flattening of the density
scrape off layer profile at high density
\cite{LaBombard:2001ks,Carralero:2015gu,Militello:2016hk,Vianello:2017ku}.
Preliminary investigations suggested that similar inter-ELM SOL
density profile broadening is observed in H-mode as well
\cite{Muller:2015jt,Carralero:2017gb,vianello:nf2019}, more pronounced
with high neutral pressure \cite{vianello:nf2019}.

The present contribution will report results obtained in a
coordinated effort within 3 different devices, JET, ASDEX-Upgrade and
TCV focusing on the SOL profile evolution in different divertor recycling
states, obtained at different recycling divertor condition, trying to
correlate the observed profile modification with different turbulent SOL
plasma transport. The mechanism of shoulder formation and the role of filamentary
transport have been tested against variation of plasma current,
magnetic topology, by comparing single and double null plasmas,
and divertor neutral densities, through modification of
cryopump efficiency.


\begingroup
\setstretch{0.8}
{\footnotesize\textbf{Acknowledgment}\\
This work has been carried out within the framework of the EUROfusion Consortium and has received funding from the Euratom
research and training programme 2014-2018 under grant agreement No 633053. The views and opinions expressed herein do not
necessarily reflect those of the European Commission.}
\begin{multicols}{2}
\setlength\bibitemsep{0pt}
\printbibliography[heading=none]
\end{multicols}
\endgroup

\end{document}

% \begin{wrapfigure}{l}{67mm}
% \centering
% \includegraphics[width=67mm]{../pdfbox/EfoldBlobAllColor.pdf}
% \caption{Top: Density decay length as a function of blob size
%   at three level of currents at constant B$_t$ in
%   AUG. Bottom: same for TCV}
% \vspace{-2.6ex}
% \label{fig:fig1}
% \end{wrapfigure}

% \begin{wrapfigure}{l}{96mm}
% \centering
% \includegraphics[width=96mm]{/Users/vianello/Desktop/Topic-21/Experiments/AUG/analysis/pdfbox/UpstreamDivertorProfiles34276_34278_34281}
% \caption{ Shots
%   \# 34276 and shot \# 34278 have the same fueling and seeding levels,
%   but the shots have been operated respectively without and with
%   the cryopump. Additional fueling and seeding have been added to Shot
%   \#34281 where cryopump was in operations. The top
%   panel show the time traces of edge density. The second rows show the
% upstream profiles normalized to the value at the separatrix for the
% three shots in 3 time instants indicated in the top panel with
% different colors. The third rows report the target profiles at
% the same instants whereas the last rows show the value of divertor
% normalized collisionality $\Lambda_{div}$ as defined \cite{Carralero:2015gu}}
% \vspace{-2ex}
% \label{fig:fig2}
% \end{wrapfigure}
